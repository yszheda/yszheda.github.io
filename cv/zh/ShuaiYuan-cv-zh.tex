% This is a redistributed latex code of LaTeX Curriculum Vitae Template
% Shuai Yuan, CS, zju & nthu
%
% Thanks: Copyright (C) 2004-2008 Jason Blevins <jrblevin@sdf.lonestar.org>
% http://jblevins.org/projects/cv-template
%
% You may use use this document as a template to create your own CV
% and you may redistribute the source code freely. No attribution is
% required in any resulting documents. I do ask that you please leave
% this notice and the above URL in the source code if you choose to
% redistribute this file.
\documentclass[letterpaper]{article}
\usepackage[BoldFont, SlantFont, CJKchecksingle]{xeCJK}
\usepackage{hyperref}
\usepackage{geometry}

\setCJKmainfont[Mapping=tex-text]{WenQuanYi Micro Hei}
\setCJKsansfont[Mapping=tex-text]{WenQuanYi Micro Hei}
\setCJKmonofont[Mapping=tex-text]{WenQuanYi Micro Hei Mono}
% Uncomment the following lines to use the Palatino font.  Remove the
% [osf] bit if you don't like the old style figures.
%
% \usepackage[T1]{fontenc}
% \usepackage[osf]{mathpazo}
% Set your name here
\def\name{袁帅}
% The following metadata will show up in the PDF properties
\hypersetup{
  colorlinks = true,
  urlcolor = black,
  pdfauthor = {\name},
  pdfkeywords = {computer science and technology},
  pdftitle = {\name: Curriculum Vitae},
  pdfsubject = {Curriculum Vitae},
  pdfpagemode = UseNone
} \geometry{textheight=8.5in, textwidth=6in}
% Customize page headers
\pagestyle{myheadings} \markright{\name} \thispagestyle{empty}
% Customize section headings
\usepackage{sectsty}
\subsectionfont{\rmfamily\mdseries\itshape\large}
% Don't indent paragraphs.
\setlength\parindent{0em}
% Make lists without bullets
% \renewenvironment{itemize}{
%   \begin{list}{}{
%     \setlength{\leftmargin}{1em}
%   }
% }{
%   \end{list}
% }
\begin{document}
\centerline{\huge\bf \name} \vspace{0.25in}
\begin{minipage}[t]{0.8\textwidth}
台湾国立清华大学资讯工程系 \\
% 手机:0988473989/(886)988473989   \\
手机: (0886)988473989/(86)13539623264   \\
Email: \href{mailto:yszheda@gmail.com}{\tt yszheda@gmail.com}\\
个人主页:\url{http://yszheda.github.io} \\
技术博客:\url{http://galoisplusplus.gitcafe.com}
\end{minipage}

\section*{教育背景}
\begin{itemize}
\item 台湾国立清华大学资讯工程专业(Computer Science)硕士,2012--2014 (GPA: 4.23/4.3).
%    \begin{itemize}
%    \item \textit{Overall GPA}: Your GPA
%    \item \textit{Ranking}: Your Ranking
%    \item \textit{Honors}: Your scholarship, year.
%    \end{itemize}
\item 浙江大学计算机科学与技术专业学士,2008--2012 (GPA: 3.84/4.0).
%    \begin{itemize}
%    \item \textit{Overall GPA}: Your GPA
%    \item \textit{Ranking}: Your Ranking
%    \item \textit{Honors}: Your scholarship, year.
%    \end{itemize}
\end{itemize}

\section*{获奖情况}
\begin{itemize}
  \item 学业优秀二等奖学金及优秀学生二等奖学金, 2008--2009
  \item 学业优秀三等奖学金及优秀学生三等奖学金, 2009--2010
  \item 学业优秀二等奖学金及优秀学生二等奖学金, 2010--2011
  \item 鸿海陆生奖学金, 2012--2013				
  \item 浙江省大学生高等数学(微积分)竞赛二等奖, 2009
  \item 浙江大学ACM竞赛三等奖, 2009
  \item 浙江大学-Intel嵌入式知识竞赛二等奖, 2010
  \item 浙江大学优秀本科毕业论文, 2012
\end{itemize}

% \section*{研究领域}
% Cloud Storage System, Erasure Codes.
% \section*{Research Interest}
% Your Interest (2-5 fileds)

\section*{研究经历}

\begin{itemize}
% \item Lab member of Eagle Lab VIPA(Visual Interaction and Graphics) group (also named ``Microsoft Visual Perception Laboratory of Zhejiang University''), 2010--2012
\item 国立清华大学LSA(Large-scale System Architecture)实验室成员,研究方向为erasure codes和cloud storage system,2012年至今
\item 法国巴黎十一大学LRI(Laboratoire de Recherche en Informatique)实验室研究实习生,研究课题为“数据库完整性约束的自动化验证”,2011,10--2012,4
%    \begin{itemize}
%	  % Rewrite in 2013
%	  \item Work on ``automated constraint verification for databases'' under the guidance of Prof.V\'eronique Benzaken and Prof. \'Evelyne Contejean.
%%	  \item Current research focuses on ``Automated constraints verification for databases''.
%	  \item Based on the observation that currently no real database management system (DBMS) have fully support the management of integrity constraints and run-time checking is time-consuming, we have present a compile-time verification strategy based on the weakest precondition and predicate transformer approaches.
%%		a strategy to verify the integrity constraints of databases at compile-time. 
%%		Our method is based on the weakest precondition and predicate transformer approaches. 
%		% Commented in 2013
%%	  \item With the help of the software verification platform Why3\href{http://why3.lri.fr/}{\tt (http://why3.lri.fr/)}, we implement integrity constraints checking for databases. All the process is fully automatic.
%    \end{itemize} 
\item 浙江大学-微软联合实验室,研究课题为“图像场景音频识别”,2010--2012
%    \begin{itemize}
%	% rewrite in 2013	
%	  \item Work on ``scene audio recognition of images'' under the supervison of Prof. Mingli Song. 
%% details
%% We apply Probabilistic Latent Semantic Analysis (pLSA) and matching pursuit (MP) algorithms to extract the features of training images and sounds respectively. Then machine learning approach is used to find the corresponding environmental sounds for a newly-input image.
%
%
%%    \item Make a project of scene audio recognition of image. Probabilistic Latent Semantic Analysis (pLSA) and matching pursuit (MP) algorithms are applied to extract the features of training images and sounds respectively. Then machine learning approach is used to find the corresponding environmental sounds for the specified image.
%%	\item  Read papers in the wide area of Speech-driven facial animation, Speech emotion recognition, AED (Audio event detection), Music emotion recognition, Sound localization, Unstructured audio scene recognition and also Image inpainting and Image completion.
%    \end{itemize}
% \item Project leader of the regular expression search/match/replace engine, which belongs to the SRTP (Student Research Train Program), 2010,4--2011,5
%     \begin{itemize}
% 	\item This project was awarded as outstanding SRTP.
% 	% \item The executable file and source code of the project is now available in Google code	\\
% 	%   \href{http://code.google.com/p/regex-engine/}{\tt (http://code.google.com/p/regex-engine/)}
% 	% \item  Learn the theory and grammar of regular expression and implement templates of regex (like regex of IP address). Use C++ boost::regex to implement the engine, which supports multi-grammars (Perl, Sed and POSIX standard regex grammar, \textmd{etc.}).
% 	% \item Take advantage of SE (Software Engineering) and PM (Program Management) knowledge to make schedule and organize progress of the project.
%     \end{itemize} 
\end{itemize}

%\section*{Academic Experience}
%\begin{itemize}
%\item \emph{Your TA/RA University}
%    \begin{itemize}
%    \item Research/Teaching  Assistant,
%     for Prof. XXX,
%     Year.
%    \end{itemize}
%\end{itemize}

%\section*{Publications}
%\begin{itemize}
%\item Publication 1
%\item Publication 2
%\end{itemize}
%\section*{Working Papers}
%\begin{itemize}
%\item Working Paper 1
%\item Working Paper 2
%\end{itemize}
%\section*{Work in Progress}
%\begin{itemize}
%\item Paper in Preparation 1
%\item Paper in Preparation 2
%\end{itemize}
%% \subsection*{Papers Under Review}
%% \subsection*{Publications in Refereed Journals}

%\section*{Computer Skills}
%Matlab, Eviews, Stata, \LaTeX, and all software you know.

\section*{项目经历}
\begin{itemize}
  \item GPU-RSCode(2012,12--2013,3):Reed-Solomon code的GPGPU加速。
	\begin{itemize}
	  \item 采用CUDA C所写。项目源代码开源并托管在Github上:\\
			  \url{https://github.com/yszheda/GPU-RSCode}
	\end{itemize}
  \item sim-outorder-extend(2013,3--2013,5):SimpleScalar sim-outorder体系结构模拟器拓展。
	\begin{itemize}
	  \item 采用C所写。项目源代码开源并托管在Github上:\\
			  \url{https://github.com/yszheda/sim-outorder-extend}
	\end{itemize}
  \item assertion-verification(2011,9--2012,3):``数据库约束的自动化验证''课题的实现。
	\begin{itemize}
	  \item 采用Ocamllex和Ocamlyacc所写。项目源代码开源并托管在Github上:\\
			  \url{https://github.com/yszheda/assertion-verification}
	\end{itemize}
  \item E-go(2011,3--2011,6):``易购''电子购物网站。
	\begin{itemize}
	  \item 采用JSP/Servlet所写,本人负责商品查询、排序及用户信息管理模块。项目源代码开源并托管在Google Code上:\\
			  \url{https://code.google.com/p/e-go/}
	\end{itemize}
  \item regex-engine(2010,10--2011,5):字符串(匹配、查找、替换)引擎。
    \begin{itemize}
	  \item 采用C++ Boost::regex库所写,本人担任组长及主要开发者。项目源代码开源并托管在Google Code上:\\
			  \url{http://code.google.com/p/regex-engine/}
      \item 该项目被评为浙江大学计算机学院第十三期大学生科研训练计划(SRTP)优秀项目。
%	  \item This project was part of the Student Research Train Program (SRTP) of Zhejiang University in 2010, and was awarded as outstanding SRTP.
%%     \item Project leader of the regular expression search/match/replace engine, which belongs to the SRTP (Student Research Train Program), 2010,4--2011,5
%%	  \item This project was awarded as outstanding SRTP.
%	  \item My role: a team leader and a programmer.
%	  \item Written in C++ Boost::regex. Source code and executable files are available on Google Code:
%
%	  \href{http://code.google.com/p/regex-engine/}{\tt (http://code.google.com/p/regex-engine/)}
	\end{itemize}

\end{itemize}

% For more projects, please refer to my profile on:
其他开源项目请访问:
\begin{itemize}
%  \item \href{http://github.com/yszheda}{\tt Github} 
%  \item \href{https://code.google.com/u/106717879882759479741/}{\tt Google Code}
  \item Github主页:\url{http://github.com/yszheda} 
  \item Google code主页:\url{https://code.google.com/u/yszheda@gmail.com/}
\end{itemize}

\section*{技能}
\begin{itemize}
% \item Language: English, Native Mandarin
\item 编程语言:C, C++, Java, Matlab/Octave, Verilog HDL, Shell script(主要为bash), Ocaml
% Assembly, OCaml(partial), ocamlyacc, ocamllex, etc.
\item 编程技术:并行计算(Hadoop MapReduce, MPI, CUDA), OpenGL等
\item 操作系统:GNU/Linux(目前使用Arch Linux), Windows
\item 版本控制工具:主要使用git,使用过svn和cvs
\item 调试工具:gdb		
\item IDE:Eclipse, Visual Studio, Xilinx ISE(目前较多项目没有用IDE,而是写Makefile或用automake)
\item 编辑器:Vim
% \item Applications: vim, Eclipse, Visual Studio, Xilinx ISE, svn, etc.
\item 文档工具:\LaTeX
% \item Documentation: \LaTeX, MS Office
\end{itemize}

% \section*{Extracurricular Activities}
% \begin{itemize}
% \item Volunteer
%     \begin{itemize}
%     \item volunteer teacher: teach some members of Chinese People's Armed Police, who are graduated from junior middle school or primary school but eager to enter the Advanced Police School, and help them with their studies.
%     \item assistant of library: put books in order and make the environment tidy.
%     \item volunteer of the energy-conserving and environment-protective activity: popularize envi-
% ronmental knowledge on the Hangzhou Canal Square.
% 	\item volunteer assistant senior: guide the entering freshmen with their admission procedure
% and class organization, also help them to adjust to the new university life.
%     \end{itemize}
% \item Literary and art activities
%    \begin{itemize}
%     \item the Mid-autumn Festival Celebrating Party of Computer Science and Technology College
% and Software Engineering College, 2008: take part as a violin performer.
%     \item the Mid-autumn Festival Celebrating Party of ChaoShan Residents' Association of Hangzhou,
% 2008: take part as a violin performer.
%     \item the Graduate Students' Admission Ceremony of Software Engineering College, 2008: take
% part as a violin performer.
% 	\item the Closing Ceremony of YuFeng Cultural Festival, 2008: take part as a violin performer.
%   \end{itemize}
% \item Social practice
%     \begin{itemize}
%     \item Social practice around Thousand Island Lake, 2009: survey the development of tourism, environmental conditions of the tourist site and environment-protective measures of the local government, and also popularize water resources protective knowledge in town and in the village nearby.
%     \end{itemize}
% \end{itemize}
% 
% 
% \section*{Interests}
% \begin{itemize}
% \item violin, Classical music, badminton, hiking, table tennis
% \end{itemize}

\bigskip
% Footer
\begin{center}
\begin{footnotesize}
%Last updated: \today \\
\end{footnotesize}
\end{center}
\end{document}
